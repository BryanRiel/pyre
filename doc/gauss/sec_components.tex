% -*- LaTeX -*-
% -*- coding: utf-8 -*-
%
% michael a.g. aïvázis
% california institute of technology
% (c) 1998-2012 all rights reserved
%

\section{Turning classes into components}
\label{sec:components}

The solutions presented in \secref{classes:generators} and \secref{classes:functors} represent
several improvements over our original attempt in \secref{simple:python}: the problem has been
decomposed into distinct parts that can evolve independently, there is a natural correspondence
with the mathematical concepts in \eqref{mc}, and the pieces are assembled together in a
natural and simple way in \secref{classes:driver:final}.

The abstract base classes do not play a very strong role in dynamically typed languages such
as python. In strongly typed languages they become constraints on their subclasses.

Changes involve editing the script and there is little checking that new objects that may be
introduced can play compatible roles. We have to wait for exceptions to get thrown to detect
improper implementations. More testing is required to compensate for the checks that the
compiler is unable to perform.

Components

%
\python{
  firstnumber=9,
  linerange={9-33},
  label={lst:components:cloud},
  caption={\srcfile{gauss/meshes/PointCloud.py}:},
}{../../examples/gauss.pyre/gauss/meshes/PointCloud.py}
%
\python{
  firstnumber=9,
  linerange={9-41},
  label={lst:components:mersenne},
  caption={\srcfile{gauss/meshes/Mersenne.py}:},
}{../../examples/gauss.pyre/gauss/meshes/Mersenne.py}
%

%
\python{
  firstnumber=9,
  linerange={9-39},
  label={lst:components:shape},
  caption={\srcfile{gauss/shapes/Shape.py}:},
}{../../examples/gauss.pyre/gauss/shapes/Shape.py}
%
%
\python{
  firstnumber=9,
  linerange={9-68},
  label={lst:components:box},
  caption={\srcfile{gauss/shapes/Box.py}:},
}{../../examples/gauss.pyre/gauss/shapes/Box.py}
%
%
\python{
  firstnumber=9,
  linerange={9-69},
  label={lst:components:ball},
  caption={\srcfile{gauss/shapes/Ball.py}:},
}{../../examples/gauss.pyre/gauss/shapes/Ball.py}
%

%
\python{
  firstnumber=9,
  linerange={9-32},
  label={lst:components:functor},
  caption={\srcfile{gauss/functors/Functor.py}:},
}{../../examples/gauss.pyre/gauss/functors/Functor.py}
%
%
\python{
  firstnumber=9,
  linerange={9-35},
  label={lst:components:constant},
  caption={\srcfile{gauss/functors/Constant.py}:},
}{../../examples/gauss.pyre/gauss/functors/Constant.py}
%
%
\python{
  firstnumber=9,
  linerange={9-58},
  label={lst:components:gaussian},
  caption={\srcfile{gauss/functors/Gaussian.py}:},
}{../../examples/gauss.pyre/gauss/functors/Gaussian.py}
%

%
\python{
  firstnumber=9,
  linerange={9-41},
  label={lst:components:integrator},
  caption={\srcfile{gauss/integrators/Integrator.py}:},
}{../../examples/gauss.pyre/gauss/integrators/Integrator.py}
%
%
\python{
  firstnumber=9,
  linerange={9-57},
  label={lst:components:montecarlo},
  caption={\srcfile{gauss/integrators/MonteCarlo.py}:},
}{../../examples/gauss.pyre/gauss/integrators/MonteCarlo.py}
%

% end of file 
