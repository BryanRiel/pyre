% -*- LaTeX -*-
% -*- coding: utf-8 -*-
%
% michael a.g. aïvázis
% orthologue
% (c) 1998-2017 all rights reserved
%

\section{Monte Carlo integration: a bit of theory}
\label{sec:montecarlo}

Suppose you have a function $f(x)$ that is sufficiently well behaved for
all $x$ in a region $\Omega \subset \mathbb{R}^{n}$ and you wish to compute the
definite integral
%
\begin{equation}
  I_{\Omega} (f)
  =
  \int_{\Omega} f
\label{eq:integral}
\end{equation}
%
The Monte Carlo integration method estimates the value of a definite integral by sampling
the function at points $x$ in $\Omega$ that are chosen at random with uniform
probability. Suppose that you have $N$ such points forming a sample $X_{N}$. The Monte
Carlo estimate for the integral is
%
\begin{equation}
  I_{\Omega} (f; X_{N})
  =
  \Omega \cdot \langle f \rangle
  =
  \Omega \, \frac{1}{N} \sum_{x \in X_{N}} f(x)
\label{eq:mc-estimate}
\end{equation}
%
where $\langle f \rangle$ is the sample mean of the function $f$ and we have used $\Omega$
as short-hand for the volume of the associated region. More details can be found in
\citep{hammersley,ueberhuber}; see \cite{weinzierl} for an excellent pedagogical
introduction to the subject.

One can show that the error in the estimate descreases as $1/\sqrt{N}$, a convergence rate that
is rather slow. It is possible to improve on this by being smart about how the region of
integration is sampled; see \cite{lepage-78,lepage-80,press}. On the bright side, the
convergence rate is {\em independent} of the dimensionality of the integral, making this method
very well suited for multi-dimensional integrals. Further, it is rather straightforward to
write a parallel implementation and compensate for the slow convergence by computing on a large
machine.

We are ready to recast \eqref{mc-estimate} in a form better suited for a computer
implementation, where we use pseudo-random number generators to create the sample set. Most
generators produce numbers between $0$ and $1$, so actual calculations require finding a box
$B_{\Omega}$ that contains $\Omega$, and building $n$-dimensional sampling points $x$ by
stretching and translating the unit interval to match the dimensions of $B_{\Omega}$. The
integration is then restricted to $\Omega$ by introducing a function
%
\begin{equation}
  \Theta_{\Omega}(x)
  =
  \left\{
  \begin{array}{ll}
  1 & x \in \Omega \\
  0 & {\rm otherwise}
  \end{array}
  \right.
\label{eq:theta}
\end{equation}
%
that takes the value $1$ inside $\Omega$ and vanishes identically outside. We can now recast
\eqref{integral} as
%
\begin{equation}
  I_{\Omega} (f)
  =
  \int_{B_{\Omega}} \Theta_{\Omega} \, f
\label{eq:integral-box}
\end{equation}
%
There are now two classes of points in the sample $X_{N}$: those interior to $\Omega$, and the
rest. Let $\tilde{N}$ be the number of points in $X_{N}$ that fall in $\Omega$; then
\eqref{mc-estimate} becomes:
%
\begin{equation}
  I_{\Omega} (f; X_{N})
  =
  \Omega \, \frac{1}{\tilde{N}} \sum_{x \in X_{\tilde{N}}} f(x)
\label{eq:mc-box}
\end{equation}
%
where $X_{\tilde{N}}$ is the subset of the sample in $\Omega$. Now, let $B$ be the volume
of the sampling box and observe that
%
\begin{equation}
  \Omega
  =
  \frac{\tilde{N}}{N} B
\label{eq:volume-estimate}
\end{equation}
%
is a good estimate of the volume of the integration region. Further, the sum over
$X_{\tilde{N}}$ in \eqref{mc-box} can be extended to $X_{N}$ as long as $f$ is
multiplied by $\Theta_{\Omega}$. We obtain
%
\begin{equation}
  I_{\Omega} (f; X_{N})
  =
  B \, \frac{1}{N} \sum_{x \in X_{N}} \Theta_{\Omega} \, f(x)
\label{eq:mc}
\end{equation}
%

In the remainder of this article, we will transform this rather innocuous expression into
a sequence of computer programs of increasing complexity and, hopefully, flexibility. The
goal is to construct a piece of software that will enable our end users to explore the
method with as little programming on their part as possible.

% end of file
